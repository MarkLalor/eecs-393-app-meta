%Copyright 2014 Jean-Philippe Eisenbarth
%This program is free software: you can 
%redistribute it and/or modify it under the terms of the GNU General Public 
%License as published by the Free Software Foundation, either version 3 of the 
%License, or (at your option) any later version.
%This program is distributed in the hope that it will be useful,but WITHOUT ANY 
%WARRANTY; without even the implied warranty of MERCHANTABILITY or FITNESS FOR A 
%PARTICULAR PURPOSE. See the GNU General Public License for more details.
%You should have received a copy of the GNU General Public License along with 
%this program.  If not, see <http://www.gnu.org/licenses/>.

%Based on the code of Yiannis Lazarides
%http://tex.stackexchange.com/questions/42602/software-requirements-specification-with-latex
%http://tex.stackexchange.com/users/963/yiannis-lazarides
%Also based on the template of Karl E. Wiegers
%http://www.se.rit.edu/~emad/teaching/slides/srs_template_sep14.pdf
%http://karlwiegers.com

%Modified for EECS 393 project in Spring 2017
%Dina Benayad-Cherif, Jason Dong, Mark Lalor, Yousef Mahmoud, Vimig Socrates
\documentclass{scrreprt}
\usepackage{listings}
\usepackage{underscore}
\usepackage{xcolor}
\usepackage[bookmarks=true]{hyperref}
\usepackage[utf8]{inputenc}
\usepackage[english]{babel}

\hypersetup{
    bookmarks=false,
    pdftitle={Software Requirements Specification},
    pdfauthor={Dina Benayad-Cherif, Jason Dong, Mark Lalor, Yousef Mahmoud, Vimig Socrates},
    pdfsubject={EECS 393 Software Requirements Specification},
    pdfkeywords={SRS, Software, Requirements, Specification},
    colorlinks=true,        % false: boxed links; true: colored links
    linkcolor=blue,         % color of internal links
    citecolor=black,        % color of links to bibliography
    filecolor=black,        % color of file links
    urlcolor=purple,        % color of external links
    linktoc=page            % only page is linked
}

\addto\captionsenglish{\renewcommand{\contentsname}{Table Of Contents}}
\def\myversion{1.0}
\date{}

\usepackage{hyperref}
\begin{document}

% Title Page
\begin{flushright}
	\rule{16cm}{5pt}\vskip1cm
	\Huge{SOFTWARE REQUIREMENTS\\ SPECIFICATION}\\
	\vspace{2cm}
	for\\
	\vspace{2cm}
	Class Collaboration Application (CCA)\\
	\vspace{2cm}
	\LARGE{Release 1.0\\}
	\vspace{2cm}
	Prepared by Dina Benayad-Cherif, Jason Dong, Mark Lalor, Yousef Mahmoud, Vimig Socrates\\
	\vfill
	\rule{16cm}{5pt}
\end{flushright}

\tableofcontents

\section*{Revision History}
All revisions of the SRS can be seen in the Git commit history of the application's meta repository.

%%%%%%%%%%%%%%%%
% INTRODUCTION %
%%%%%%%%%%%%%%%%
\chapter{Introduction}
\section{Purpose}
This SRS outlines both, the functional and nonfunctional requirements for the first release of Class Collaboration Application (CCA). Overall, this document is meant for the project members that will fully develop CCA. All specifications here, unless otherwise specified, are required for the first release.
\section{Project Scope }
The application will allow university students to augment their learning in coursework. The application seeks to connect students through their courses by providing a platform to discuss assignments, share notes, and communicate with one another.
\section{References}
\begin{itemize}
	\item \textit{Google App Engine Python Documentation}. \\
	\url{https://cloud.google.com/appengine/docs/python}
	\item Podgurski, Andy. \textit{EECS 393 Lectures. Spring 2017}.
	\item Wiegers, Karl. \textit{Cafeteria Ordering System (COS) SRS document. November 4, 2002}.
	\item Wiegers, Karl. \textit{COS Vision and Scope document. November 4, 2002}.
\end{itemize}

%%%%%%%%%%%%%%%%%%%%%%%
% OVERALL DESCRIPTION %
%%%%%%%%%%%%%%%%%%%%%%%
\chapter{Overall Description}
\section{Product Perspective}
Our app allows users to join an ecosystem in which students can share notes, discuss solutions, and plan projects and study sessions. This app will target Case Western students (keeping in mind being able to target general college students later). The app creates a way for individuals in a class to connect with his/her peers. The problem with classroom collaboration today is that if you are in a class without friends you are at a huge disadvantage. With our app however the entire class will be able to work together and benefit from each other’s knowledge without having to have a deep connection between classmates.
\section{Classes and Characteristics}
\subsection{Student}
A Student is a Case Western Reserve University Student who would like to collaborate with other classmates through chat and document sharing. Students will look to find documents uploaded by other classmates from previous years as well as discuss assignments relevant to their classes.
\subsection{Class}
A Class is a course offered by Case Western Reserve University. Courses usually give out a varying number of assignments, quizzes, and exams that students are constantly preparing for.
\subsection{ClassItem}
A ClassItem is a manager for a particular entity that a Student might create. These could be a quiz/exam, homework, or notes. 
Each ClassItem attribute of a ClassItem that defines whether the document is a quiz, homeworks, notes. 
\subsection{Document}
A Document is a resource uploaded by a user. It can be associated with a ClassItem for organizational purposes. It can be any permitted file type, such as a scanned PDF of one’s notes, a Mathematica file, or Java source code. 
\subsection{Chat}
A Chat is a real-time discussion forum for a general course or a particular ClassItem. It can be used to discuss the attached documents or the quiz, exam, or assignment associated with it.

\section{Operating Environment}
\begin{itemize}
	\item The Class Collaboration Application will be able to operate on any modern web browser.
	\item Google App Engine will be managing our servers.
	\item Users will have access to the application if they are currently Case students with a Case ID.
	\item Users may access the application from any location (not limited to the Case network).
\end{itemize}

\section{Design and Implementation Constraints}
\begin{itemize}
	\item Scripts and application will be written in Python on the Google App Engine platform.
	\item Google Cloud Datastore (schemaless) / Google Query Language will be used to store data.
	\item User authentication will be handled by Google App Engine.
	\item Python code will follow PEP8 Style Guide.
	\item HTML code shall conform to the HTML 5 standard.
\end{itemize}

\section{User Documentation}
\begin{itemize}
	\item Features should be intuitive and minimal enough that users can navigate and interact with the web app without a tutorial.
	\item The application will contain a FAQ and/or help page that will provide help to the users with common problems.
\end{itemize}

\section{Assumptions and Dependencies}
\begin{itemize}
	\item AS: Users are Case Western Reserve University students.
	\item DE: The application’s content is defined by the current users. Only courses that have been added by users will appear. The system does not directly integrate with Case Western’s databases to find out what courses exist.
\end{itemize}

%%%%%%%%%%%%%%%%%%%
% SYSTEM FEATURES %
%%%%%%%%%%%%%%%%%%%
\chapter{System Features}

% Authentication and User Information %
\section{Authentication and User Information}
\subsection{Description}
Users will log in with a Google account ending in @case.edu. An account will be created using this authentication. Canvas’ API will be used to identify a user's courses and automatically ask the user to add courses they are in. Every user will have relevant information available in the form of a user profile.
\subsection{Stimulus/Response Sequences}
\begin{tabular}{| p{2cm} | p{12cm} | }
	\hline
	Stimulus:   & User logs in for the first time with a Google account ending in @case.edu. \\
	Response:   & Prompts user to fill out basic account information, creates a new user profile in database. Also asks them to authenticate with Canvas to automatically suggest classes. \\ \hline
	Stimulus:   & User logs in not for the first time. \\
	Response 1: & Load personalized home page. \\
	Response 2: & Check Canvas for new assignments. \\ \hline
	Stimulus:   & User edits his/her profile information or joins more courses. \\
	Response:   & Update the database with his/her information. \\
	\hline
\end{tabular}

\subsection{Functional Requirements}

\begin{tabular}{| p{3cm} | p{11cm} |}
\hline
UserManager.New & Creates a new user in the database granted that they are able to authenticate with Google with a *@case.edu email address. \\ \hline
Users.SetName & Allows user to change their first and/or last name. \\ \hline
Users.SetClasses & Allows user to modify the classes they are subscribed to. \\ \hline
Users.GetClasses & Gets classes a user is subscribed to. \\
\hline
\end{tabular}

% Communication Channels %
\section{Communication Channels}
\subsection{Description}
Users who have joined specific course pages will have access to communication channels in which all users within that course will be able to communicate with each other. Communication channels will include a general chat for each course page and a chat for each particular assignment and resource that users can discuss.

\subsection{Stimulus/Response Sequences}
\begin{tabular}{| p{2cm} | p{12cm} | }
	\hline
	Stimulus: & User joins a course page. \\
	Response: & User is granted access to communication channels specific specific to that page. \\ \hline
	Stimulus: & User sends a message to a communication channel on course page. \\
	Response: & User’s message delivered to all other users with access to the channel. \\ \hline
	Stimulus: & User creates a new note, uploads a document, or creates an assignment. \\
	Response: & Channel is created for that resource. \\
	\hline
\end{tabular}

\subsection{Functional Requirements}
\begin{tabular}{| p{4cm} | p{10cm} |}
\hline
User.SelectCoursePage & User is given access to all communication channels within selected course page. \\ \hline
User.CreateChannel & Channel is created for the particular resource within the selected course. \\ \hline
User.SelectChannel  & User’s chat area is replaced with the newly selected channel. \\ \hline
User.SendMessage & User sends message to currently selected channel. \\
\hline
\end{tabular}


% Documents and Note Sharing %
\section{Documents and Note Sharing}
\subsection{Description}
Users will be able to upload images, pdfs, or documents of their notes from the class that is accessible to other individuals in the same classes. Any end user who uploaded any document to CCA will be able to edit or remove the document. 
\subsection{Stimulus/Response Sequences}
\begin{tabular}{| p{2cm} | p{12cm} | }
	\hline
	Stimulus: & The end user requests to upload a their class notes or course documents. \\
	Response: & Blob created by uploading a file through an HTTP request, where Blobstore creates a blob from the file’s contents and returns blob reference. \\ \hline
	Stimulus: & The end user requests to view a file uploaded. \\
	Response: & Blobstore will return a blob_reader from the blob_key that will return the original file to the end user. \\ \hline
	Stimulus: & The end user removes a file they uploaded.\\
	Response: & Blobstore will use the blob_key to delete the correlating file from the storage system.\\
	\hline

\end{tabular}

\subsection{Functional Requirements}

\begin{tabular}{| p{4.5cm} | p{9.5cm} |}
\hline
Document.GetUploader & Application allows access to see who posted a specific document. \\ \hline
Document.DeleteUpload & Application allows end user to delete a document they uploaded themselves. If the user is not the one who uploaded a file, cannot delete it from the system. \\ \hline
Document.UploadDate & Retrieves the date that the document was uploaded. \\ \hline
Document.GetMetadata & Retrieves all the metadata that was entered by the original uploader. Must include class, may optionally (recommended) include date taken and optionally section / chapter number. \\ \hline
DocumentManager.Upload & Application allows end user who is logged into the application to upload a document. \\ \hline
DocumentManager.Get & Gets a list of documents for the specified class (can be refined to a certain date or section range). \\

\hline
\end{tabular}

%%%%%%%%%%%%%%%%%%%%%%%%%%%%%%%%%%
% EXTERNAL INTERFACE REQUIRMENTS %
%%%%%%%%%%%%%%%%%%%%%%%%%%%%%%%%%%
\chapter{External Interface Requirements}
\section{User Interfaces}
\begin{itemize}
	\item UI-1: We will be using Bootstrap to build our user interface for CCA.
	\item UI-2: There will be a Help page that will advise the user on how to use certain features of CCA.
	\item UI-3: The system will only allow for mouse and keyboard combinations for data entry.
\end{itemize}

\section{Hardware Interfaces}
No Hardware Interfaces have been identified.

\section{Software Interfaces}
\begin{itemize}
\item SI-1: Canvas will be used to recommend courses for a user to join and create assignments within existing classes.
\end{itemize}

\section{Communications Interfaces}
\begin{itemize}
	\item CI-1: The CCA will contain an internal messaging system that can be accessed by users through the web application.
	\item CI-2: The CCA will send out emails when a user is mentioned in a chat or added to a new chat.
\end{itemize}
%%%%%%%%%%%%%%%%%%%%%%%%%%%%%%%%%%%
% OTHER NONFUNCTIONAL REQUIRMENTS %
%%%%%%%%%%%%%%%%%%%%%%%%%%%%%%%%%%%
\chapter{Other Nonfunctional Requirements}

\section{Performance Requirements}
\begin{itemize}
	\item PE-1: The system shall accommodate up to 100 users at all times. A majority of users will be logged in for long periods of time to send and receive messages, and finish homework, so we estimate an average session to be up to 2 hours of noncontinuous use.
	\item PE-2: Chat messages sent by users will be visible to others in others in no less than 10 seconds.
	\item PE-3: The application may be down for at most 1 hour a week for maintenance. 
\end{itemize}

\section{Safety Requirements}
No safety requirements have been identified.

\section{Security Requirements}
\begin{itemize}
	\item SE-1: Users will be required to have a Case email address and log in through the Google’s authentication system.
	\item SE-2: Users will only be able to view the general chats and assignment/resource chats for the courses they've joined.
	\item SE-3: Only users who have logged in will be allowed to upload documents to the system
\end{itemize}

\section{Software Quality Attributes}
\begin{itemize}
	\item Availability-1: The CCA will be available to users on the internet 99.5\% of the time.
\end{itemize}


\end{document}
