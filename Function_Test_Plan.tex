%Copyright 2014 Jean-Philippe Eisenbarth
%This program is free software: you can 
%redistribute it and/or modify it under the terms of the GNU General Public 
%License as published by the Free Software Foundation, either version 3 of the 
%License, or (at your option) any later version.
%This program is distributed in the hope that it will be useful,but WITHOUT ANY 
%WARRANTY; without even the implied warranty of MERCHANTABILITY or FITNESS FOR A 
%PARTICULAR PURPOSE. See the GNU General Public License for more details.
%You should have received a copy of the GNU General Public License along with 
%this program.  If not, see <http://www.gnu.org/licenses/>.

%Based on the code of Yiannis Lazarides
%http://tex.stackexchange.com/questions/42602/software-requirements-specification-with-latex
%http://tex.stackexchange.com/users/963/yiannis-lazarides
%Also based on the template of Karl E. Wiegers
%http://www.se.rit.edu/~emad/teaching/slides/srs_template_sep14.pdf
%http://karlwiegers.com

%Modified for EECS 393 project in Spring 2017
%Dina Benayad-Cherif, Jason Dong, Mark Lalor, Yousef Mahmoud, Vimig Socrates
\documentclass{scrreprt}
\usepackage{listings}
\usepackage{underscore}
\usepackage{xcolor}
\usepackage[bookmarks=true]{hyperref}
\usepackage[utf8]{inputenc}
\usepackage[english]{babel}
\usepackage{tabularx}

\hypersetup{
    bookmarks=false,
    pdftitle={Functional Test Plan},
    pdfauthor={Dina Benayad-Cherif, Yousef Mahmoud, Vimig Socrates},
    pdfsubject={EECS 393 Functional Test Plan},
    pdfkeywords={},
    colorlinks=true,        % false: boxed links; true: colored links
    linkcolor=blue,         % color of internal links
    citecolor=black,        % color of links to bibliography
    filecolor=black,        % color of file links
    urlcolor=purple,        % color of external links
    linktoc=page            % only page is linked
}

\addto\captionsenglish{\renewcommand{\contentsname}{Table Of Contents}}
\def\myversion{1.0}
\date{}

\usepackage{hyperref}
\begin{document}

% Title Page
\begin{flushright}
	\rule{16cm}{5pt}\vskip1cm
	\Huge{Functional Test Plan and Specifications}\\
	\vspace{2cm}
	for\\
	\vspace{2cm}
	Class Collaboration Application (CCA)\\
	\vspace{2cm}
	\LARGE{Release \myversion \\}
	\vspace{2cm}
	Prepared by Dina Benayad-Cherif, Yousef Mahmoud, Vimig Socrates\\
	\vfill
	\rule{16cm}{5pt}
\end{flushright}

\tableofcontents

%%%%%%%%%%%%%%%%
% INTRODUCTION %
%%%%%%%%%%%%%%%%


\chapter{Introduction}
This Test Plan version 1.0 is the testing strategy for the Class Collaboration Application. It provides insight on how we will be testing our application so that all features work properly and fail properly in all conditions. 

\section{Purpose}
This Test Plan goes over:
\begin{enumerate}
\item All project information
\item Testing types and strategies to be used
\item A testing schedule
\end{enumerate}

\section{Objectives}
This document is the testing strategy for the Class Collaboration Application (CCA) and to make sure that we meet all of our requirements specified in the Software Requirements Specifications. This plan is meant to be an outline for the individuals testing the functionality of the application and evaluating the quality of the CCA. 

\section{Scope}
This section of the Test Plan describes the scope of the project Class Collaboration Application. 

Class Collaboration Application will be tested for its functionality on any modern 	web browser and on any operating system. Testing includes, unit testing, integration testing and system testing.
 

The testing includes testing for functions:
\begin{itemize}
\item User login using Single Sign-On (SSO)
\item Creating a class
\item Joining a class
\item Uploading/downloading/removing a file
\item Class chats operating correctly
\item Updating a user profile
\end{itemize}

\chapter{Test Strategies/Techniques}
This section of the document is meant to outline the test design and test data for the Class Collaboration Application. 

\begin{enumerate}
\item Test Design
\begin{enumerate}
\item Unit Testing - A type of testing that verifies the application logic based on the program structure
\item Integration Testing - A type of testing that verifies the system’s functionality according to the Software Design Document.
\item System Testing - A type of black box testing that evaluates the system’s compliance against specified requirements.
\end{enumerate}
\end{enumerate}


\section{Test Cases}
\begin{center}
\begin{tabularx}{\linewidth}{lXX}
Use Case ID & Description & Test Case \\ \hline
UC-1 & Use Case Login & TC-1 \\ \hline
UC-2 & Use Case Update User Profile & TC-3 \\ \hline
UC-3 & Use Case Create a Class & TC-4 \\ \hline
UC-4 & Use Case Join a Class & TC-4 \\ \hline
UC-5 & Use Case Upload a Document & TC-5 \\ \hline
UC-6 & Use Case Download a Document & TC-6 \\ \hline
UC-7 & Use Case Remove a Document & TC-7 \\ \hline
UC-8 & Use Case Join a Chat & TC-8 \\ \hline
UC-9 & Use Case Send a Chat & TC-9 \\ \hline
UC-10 & Use Case Synchronize Chat & TC-10 \\ \hline
UC-11 & Use Case Login After Expired Session & TC-11 \\ \hline
\end{tabularx}
\end{center}

Table 1 above illustrates all the test cases.


\begin{tabularx}{\linewidth}{lXXX}

Test Case & Use Case ID & Description & Expected Result \\ \hline
TC-1 & UC-1 & User enters a valid Case login username and password. & User is given access to the application \\ \hline
TC-2 & UC-2 & User edits their bio information in their user profile. & A message shows whether or not the information was successfully saved. \\ \hline
TC-3 & UC-3 & If the class does not exist in the database, user can create it. & An empty class with a class chat will be populated. \\ \hline
TC-4 & UC-4 & User is taking a course that exists in the database, they can click to join. & The class will be added to the user’s current courses. \\ \hline
TC-5 & UC-5 & User uploads  a document to a course. & The document will automatically download from Blobstore and user is the Admin of that document. \\ \hline
TC-6 & UC-6 & User downloads a document in a course they are enrolled in. & The user will be able to access the document on their local machine. \\ \hline
TC-7 & UC-7 & Admin of the document uploaded removes the file from the course. & A message will display that it was successfully taken down from the course and the database. \\ \hline
TC-8 & UC-8 & User joins a class chat. & Chat will show that the User just joined in. \\ \hline
TC-9 & UC-9 & User sends a chat & Recipients will receive a chat from the user \\ \hline
TC-10 & UC-10 & Chats synchronize between all users & Chats will be displayed without the need to refresh the page \\ \hline
TC-11 & UC-11 & Session is idle for 15 minutes. & Pop up message requesting user to login to CCA again. \\ \hline
\end{tabularx}
\\
\\
\\
\\
Table 2 above describes each of the test cases mentioned in Table 1 and the expected results.
\\
\\
\begin{tabularx}{\linewidth}{lX}
Tester & Test Cases Testing \\ \hline
Vimig Socrates & TC 1-4 \\ \hline
Dina Benayad-Cherif & TC 5-7 \\ \hline
Yousef Mahmoud & TC 8-11 \\ \hline
\end{tabularx}
\\
\\
\\
Table 3 above represents who will be testing each of the specifications by March 31, 2017.

\chapter{Goals and Deliverables}
The goals for this are to accomplish all tasks described in the document, to build a solid testing process, and to verify the current state of the Class Collaboration Application.

\chapter{Inspection Report}
\begin{center}
\begin{tabular}{| p{2cm} | p{2cm} | p{6cm} | p{2cm} | p{2cm} |}
	\hline
	\multicolumn{5}{ |c| }{\Large Inspection for Functional Test Plan} \\ \hline
	\multicolumn{3}{ |p{9cm}| }{\textbf{Name of Inspectors:} Vimig Socrates, Dina Benayad-Cherif, and Yousef Mahmoud}  & \multicolumn{2}{ |p{5cm}| }{\textbf{Name of Project:} Class Collaboration Application (CCA)} \\ \hline
	\multicolumn{5}{ |l| }{\textbf{Name of Deliverable:} Functional Test Plan and Specifications} \\ \hline
		\multicolumn{3}{ |p{9cm}| }{\textbf{Inspection Date:} 3/27/2017}  & \multicolumn{2}{ |p{5cm}| }{\textbf{Reinspection:} No} \\ \hline
	Defect \# & Location & Description & Severity & Resolved \\ \hline
	1 & Throughout & Fixed the formatting of the headers & Min & Yes (Yousef) \\ \hline
	2 & Table 2 & Fixed some of the descriptions of the test cases & Maj & Yes (Dina) \\ \hline
	3 & Throughout & Fixed some of the spacing in the section. & Min & Yes (Dina) \\ \hline
	4 & Testing Strategies &Added some more test cases & Maj & Yes (Yousef) \\ \hline
	5 & Throughout & Converted full document to Latex and reviewed for consistency of test cases & Maj & Yes (Vimig) \\ \hline
\end{tabular}
\end{center}

\end{document}
