%Copyright 2014 Jean-Philippe Eisenbarth
%This program is free software: you can 
%redistribute it and/or modify it under the terms of the GNU General Public 
%License as published by the Free Software Foundation, either version 3 of the 
%License, or (at your option) any later version.
%This program is distributed in the hope that it will be useful,but WITHOUT ANY 
%WARRANTY; without even the implied warranty of MERCHANTABILITY or FITNESS FOR A 
%PARTICULAR PURPOSE. See the GNU General Public License for more details.
%You should have received a copy of the GNU General Public License along with 
%this program.  If not, see <http://www.gnu.org/licenses/>.

%Based on the code of Yiannis Lazarides
%http://tex.stackexchange.com/questions/42602/software-requirements-specification-with-latex
%http://tex.stackexchange.com/users/963/yiannis-lazarides
%Also based on the template of Karl E. Wiegers
%http://www.se.rit.edu/~emad/teaching/slides/srs_template_sep14.pdf
%http://karlwiegers.com

%Modified for EECS 393 project in Spring 2017
%Dina Benayad-Cherif, Jason Dong, Mark Lalor, Yousef Mahmoud, Vimig Socrates
\documentclass{scrreprt}
\usepackage{listings}
\usepackage{underscore}
\usepackage{xcolor}
\usepackage[bookmarks=true]{hyperref}
\usepackage[utf8]{inputenc}
\usepackage[english]{babel}

\hypersetup{
    bookmarks=false,
    pdftitle={Software Design Document},
    pdfauthor={Dina Benayad-Cherif, Jason Dong, Mark Lalor, Yousef Mahmoud, Vimig Socrates},
    pdfsubject={EECS 393 Software Design Document},
    pdfkeywords={SRS, Software, Design},
    colorlinks=true,        % false: boxed links; true: colored links
    linkcolor=blue,         % color of internal links
    citecolor=black,        % color of links to bibliography
    filecolor=black,        % color of file links
    urlcolor=purple,        % color of external links
    linktoc=page            % only page is linked
}

\addto\captionsenglish{\renewcommand{\contentsname}{Table Of Contents}}
\def\myversion{1.0}
\date{}

\usepackage{hyperref}
\begin{document}

% Title Page
\begin{flushright}
	\rule{16cm}{5pt}\vskip1cm
	\Huge{SOFTWARE DESIGN\\ DOCUMENT}\\
	\vspace{2cm}
	for\\
	\vspace{2cm}
	Class Collaboration Application (CCA)\\
	\vspace{2cm}
	\LARGE{Release \myversion \\}
	\vspace{2cm}
	Prepared by Dina Benayad-Cherif, Jason Dong, Mark Lalor, Yousef Mahmoud, Vimig Socrates\\
	\vfill
	\rule{16cm}{5pt}
\end{flushright}

\tableofcontents

%%%%%%%%%%%%%%%%
% INTRODUCTION %
%%%%%%%%%%%%%%%%

\chapter{Introduction}

\section{Purpose}
This design document describes the overall structure of the Class Collaboration Application (CCA) by outlining significant aspects of the system’s architecture.


\section{Scope}
This application will be used for students at Case Western Reserve University to collaborate and discuss specific courses they are in with other students. Some of the key features of this application are that students will be able to chat with other students in the course, upload class notes or other miscellaneous documents to discuss, and ask questions. Users will also be able to download any documents uploaded for a course that they are enrolled in and can personalize their own account on the CCA. 

\section{Acronyms, Abbreviations and Definitions}
\textit{CCA (Class Collaboration Application)} - Acronym of the name of the application.  \\
\textit{PaaS (Platform as a Service)} - Acronym of a cloud computing platform. \\
\textit{IaaS (Infrastructure as a Service)} - Acronym of a cloud computing infrastructure.\\
\textit{VM (Virtual Machine)} - Acronym of virtual machine. \\
\textit{SSO (Single Sign-On)} - Acronym of a tool for access control across several independent software systems. 


\section{Overview}

\section{Reference Material}

\begin{itemize}
	\item Book Inventory System System Design Document
	\item ZZZ Software Architecture
\end{itemize}

%%%%%%%%%%%%%%%%%%%
% SYSTEM OVERVIEW %
%%%%%%%%%%%%%%%%%%%

\chapter{System Overview}

%%%%%%%%%%%%%%%%%%%%%
% PRINCIPAL CLASSES %
%%%%%%%%%%%%%%%%%%%%%

\chapter{Principal Classes}
	
\section{}

\section{}

\section{}

%%%%%%%%%%%%%%%%%%%%%%%
% SYSTEM ARCHITECTURE %
%%%%%%%%%%%%%%%%%%%%%%%

\chapter{System Architecture}

\section{Architectural Design}

\section{Decomposition Description}

\section{Design Rationale}

%%%%%%%%%%%%%%%%%%%%
% CLASS INTERFACES %
%%%%%%%%%%%%%%%%%%%%

\chapter{Class Interfaces}

\section{Class Student}
An instance of Student represents a user who can create CourseItems and 		Courses. A student is also able to upload Documents attached to a CourseItem.
\subsection{Public Constructor Student}
\textit{Student(String name, String id, Time lastLogin)} \\
Creates a student object containing the name and Case ID.

\subsection{Public Method GetStudentID}
\textit{String GetStudentID()} \\
Returns the Case ID of the user.

\subsection{Public Method GetDurationOfSession}
\textit{Time GetDurationOfSession()} \\
Returns the length of the current login session of the user.

\subsection{Public Method IsSessionExpired}
\textit{Boolean IsSessionExpired()} \\
Returns whether the the current login session is expired.

\subsection{Public Method CreateCourseItem}
\textit{Boolean CreateCourseItem(dict options, Course course)} \\
Creates an instance of a CourseItem and returns whether it was 		successfully made.

\subsection{Public Static Method CreateCourse}
\textit{Boolean CreateCourse(String courseID, String CourseName)} \\
Creates an instance of a Course and returns whether it was			 successfully made.

\subsection{Public Method GetEnrolledCourses}
\textit{List<Course> GetEnrolledCourses()} \\
Returns the list of courses a student is currently enrolled in on the application.

\subsection{Public Method JoinCourse}
\textit{Boolean JoinCourse(Course course)} \\
Returns whether a student successfully joined to a preexisting course.

\section{Class Course}

\subsection{Public Method GetStudents}
\textit{List<Student> GetStudents()} \\
Queries the students table of the database to see which students 		have added this course to their list of courses.

\subsection{Public Method GetCourseItems()}
\textit{List<CourseItem> GetCourseItems()} \\
Queries the database to return a list of all CourseItems associated with the course.

\subsection{Public Method GetChat}
\textit{Chat GetChat()} \\
Returns the instance of the Chat class associated with the Course.

\section{Class CourseItem}
An object created by a student that can contain a document relevant to the course. This object is associated with a Course object.

\subsection{Public Method RemoveCourseItem}
\textit{Boolean RemoveCourseItem()} \\
Removes the CourseItem from the database, so that it is no longer available to Students.

\subsection{Public Method ModifyCourseItem}
\textit{Boolean ModifyCourseItem(Document document)} \\
Adds the document to the already existing CourseItem.

\section{Class Document}

\subsection{Public Method Download}
\textit{Void Download()} \\
Returns whether document was successfully downloaded to the user's machine.

\subsection{Public Method GetCourseItemID}
\textit{String GetCourseItemID()} \\
Returns the unique id associated with the course item.

\subsection{Public Method ShareDocument}
\textit{Void ShareDocument()} \\
Allows the user to share the document through email.

\section{Class Chat}
Represents the chat for a course. The Chat object will hold all ChatMessage objects for a course.

\subsection{Public Method GetChatVersion}
\textit{Integer GetChatVersion()} \\
Returns what number message the chat has currently advanced to. This is cached whenever possible and queried often by the user to know when to request updates.

\subsection{Public Method GetChatMessages}
\textit{List<ChatMessage> GetChatMessages(Integer number = 50)} \\
Returns that last \textit{number} chat messages associated with this chat (default 50).

\subsection{Public Method SendMessage}
\textit{Void SendMessage(String content, Student author)} \\
Creates a ChatMessage object with a string content and sends it to the chat.

\subsection{Public Method SendDocument}
\textit{Void SendDocument(Document doc, Student author)} \\
Creates a ChatMessage object with a document attached and sends it to the chat.

\section{Class ChatMessage}
Represents a single message sent in a Chat. This message can be text or a document.

\subsection{Public Enum MessageType}
\textit{} \\
Indicates the chat message type. Currently can be:
\begin{itemize}
	\item Text
	\item Document
	\item Image
\end{itemize}

\subsection{Public Method GetType}
\textit{MessageType GetType()} \\
Returns the type of the message.

\subsection{Public Method GetContent}
\textit{Void GetContent()} \\
Returns the content associated with the ChatMessage.

\subsection{Private Method GetText()}
\textit{String GetText()} \\
Returns the text associated with this ChatMessage

\subsection{Private Method GetURL()}
\textit{String GetURL()} \\
Returns the URL associated with this ChatMessage.

\subsection{Public Method GetStudent}
\textit{Student GetStudent()} \\
Returns the Student who created the ChatMessage object.

\subsection{Public Method GetTime}
\textit{Time GetTime()} \\
Returns the time the ChatMessage object was sent.

\subsection{Public Method Delete()}
\textit{Void Delete()} \\
Removes the ChatMessage object from the database. This object will no longer be shown in the Chat object class.

%%%%%%%%%%%%%%%%%%%%%%%%%%
% HUMAN INTERFACE DESIGN %
%%%%%%%%%%%%%%%%%%%%%%%%%%

\chapter{Human Interface Design}

\section{Overview of User Interface}

\section{Screen Images}

\section{Screen Objects and Actions}


\end{document}
